\documentclass[]{tufte-handout}

% ams
\usepackage{amssymb,amsmath}

\usepackage{ifxetex,ifluatex}
\usepackage{fixltx2e} % provides \textsubscript
\ifnum 0\ifxetex 1\fi\ifluatex 1\fi=0 % if pdftex
  \usepackage[T1]{fontenc}
  \usepackage[utf8]{inputenc}
\else % if luatex or xelatex
  \makeatletter
  \@ifpackageloaded{fontspec}{}{\usepackage{fontspec}}
  \makeatother
  \defaultfontfeatures{Ligatures=TeX,Scale=MatchLowercase}
  \makeatletter
  \@ifpackageloaded{soul}{
     \renewcommand\allcapsspacing[1]{{\addfontfeature{LetterSpace=15}#1}}
     \renewcommand\smallcapsspacing[1]{{\addfontfeature{LetterSpace=10}#1}}
   }{}
  \makeatother

\fi

% graphix
\usepackage{graphicx}
\setkeys{Gin}{width=\linewidth,totalheight=\textheight,keepaspectratio}

% booktabs
\usepackage{booktabs}

% url
\usepackage{url}

% hyperref
\usepackage{hyperref}

% units.
\usepackage{units}


\setcounter{secnumdepth}{-1}

% citations
\usepackage{natbib}
\bibliographystyle{plainnat}

% pandoc syntax highlighting

% longtable

% multiplecol
\usepackage{multicol}

% strikeout
\usepackage[normalem]{ulem}

% morefloats
\usepackage{morefloats}


% tightlist macro required by pandoc >= 1.14
\providecommand{\tightlist}{%
  \setlength{\itemsep}{0pt}\setlength{\parskip}{0pt}}

% title / author / date
\title{Make an R Cheat Sheet}
\author{Jarrett Byrnes}
\date{2020-02-27}


\begin{document}

\maketitle




\hypertarget{introduction}{%
\section{Introduction}\label{introduction}}

In the wild and wolly world of R, there are many packages out there.
Some of them are on the Comprehensive R Archive Network\footnote{See
  \href{https://cran.r-project.org/}{CRAN}}. Many can be found on
\href{http://github.com/}{Github}\footnote{A site for sharing code and
  using \texttt{git} as version control software. If you're interested,
  see \href{https://happygitwithr.com/}{Happy with Git} for more and
  talk to me. Extra credit awaits\ldots{}} and installed with the
\href{https://devtools.r-lib.org/}{devtools} package\footnote{instead of
  \texttt{install.packages} you use
  \texttt{devtools::install\_github(\textquotesingle{}username/pkgname\textquotesingle{})}}.

Throughout the semester, we've been refering you to
\href{https://rstudio.com/resources/cheatsheets/}{R Cheat Sheets} for
everything from
\href{https://github.com/rstudio/cheatsheets/raw/master/strings.pdf}{working
with strings} to
\href{https://github.com/rstudio/cheatsheets/raw/master/data-visualization-2.1.pdf}{ggplot2}
and more. These cheat sheets are invaluable as learning tools. Creating
a cheat sheet is also an amazing way to familiarize yourself with a new
package, and really solidify your knowledge of the R ecosystem.

So, for your midterm, I'd like you to create a cheat sheet for the
package of your choice. It can be a simple package, or a complex
package, or anything in between. Below, I'll detail the steps of the
process, as well as provide a list of some packages that might be
enjoyable to use for this assignment - although the choice is ultimately
up to you!

\hypertarget{steps}{%
\section{Steps}\label{steps}}

\begin{enumerate}
\def\labelenumi{\arabic{enumi}.}
\item
  First, look at existing
  \href{https://rstudio.com/resources/cheatsheets/}{R Cheat Sheets}. I
  will give you a homework question asking you to identify what you do
  and do not find useful about the examples here shortly.
\item
  Second, read up on
  \href{https://rstudio.com/resources/cheatsheets/how-to-contribute-a-cheatsheet/}{how
  to create an R Cheat Sheet}. Download the template and read it over
  CAREFULLY.
\item
  Second, select a package. It can be from the list below. It can be
  from CRAN. It might be from a
  \href{https://cran.r-project.org/web/views/}{CRAN Task View}. You
  might find it on \href{https://awesome-r.com/}{awesome-r.com}. Or take
  a look at the New and Updated packages section of
  \href{https://rweekly.org/\#NewPackages}{rweekly} (note - I read this
  daily!).
\end{enumerate}

\hypertarget{packages}{%
\section{Packages}\label{packages}}

\url{https://github.com/qinwf/awesome-R}
\url{https://ropengov.github.io/projects/}
\url{https://github.com/rOpenHealth}
\url{https://ropensci.org/packages/}



\end{document}
